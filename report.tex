%%% Template originaly created by Karol Kozioł (mail@karol-koziol.net) and modified for ShareLaTeX use

\documentclass[a4paper,11pt]{article}

\usepackage{listings} % Required for inserting code snippets
\usepackage[usenames,dvipsnames]{color} % Required for specifying custom colors and referring to colors by name


\usepackage[T1]{fontenc}
\usepackage[utf8]{inputenc}
\usepackage{graphicx}
\usepackage{xcolor}

\renewcommand\familydefault{\sfdefault}
\usepackage{tgheros}
\usepackage[defaultmono]{droidmono}

\usepackage{amsmath,amssymb,amsthm,textcomp}
\usepackage{enumerate}
\usepackage{multicol}
\usepackage{tikz}

\usepackage{geometry}
\geometry{total={210mm,297mm},
left=25mm,right=25mm,%
bindingoffset=0mm, top=20mm,bottom=20mm}


\linespread{1.3}

\newcommand{\linia}{\rule{\linewidth}{0.5pt}}

% custom theorems if needed
\newtheoremstyle{mytheor}
    {1ex}{1ex}{\normalfont}{0pt}{\scshape}{.}{1ex}
    {{\thmname{#1 }}{\thmnumber{#2}}{\thmnote{ (#3)}}}

\theoremstyle{mytheor}
\newtheorem{defi}{Definition}

% my own titles
\makeatletter
\renewcommand{\maketitle}{
\begin{center}
\vspace{2ex}
{\huge \textsc{\@title}}
\vspace{1ex}
\\
\linia\\
\@author \hfill \@date
\vspace{4ex}
\end{center}
}
\makeatother
%%%

% custom footers and headers
\usepackage{fancyhdr}
\pagestyle{fancy}
\lhead{}
\chead{}
\rhead{}
\lfoot{Programming Assignment}
\cfoot{}
\rfoot{Page \thepage}
\renewcommand{\headrulewidth}{0pt}
\renewcommand{\footrulewidth}{0pt}
%

% code listing settings
\usepackage{listings}
\lstset{
    language=Python,
    basicstyle=\ttfamily\small,
    aboveskip={1.0\baselineskip},
    belowskip={1.0\baselineskip},
    columns=fixed,
    extendedchars=true,
    breaklines=true,
    tabsize=4,
    prebreak=\raisebox{0ex}[0ex][0ex]{\ensuremath{\hookleftarrow}},
    frame=lines,
    showtabs=false,
    showspaces=false,
    showstringspaces=false,
    keywordstyle=\color[rgb]{0.627,0.126,0.941},
    commentstyle=\color[rgb]{0.133,0.545,0.133},
    stringstyle=\color[rgb]{01,0,0},
    numbers=left,
    numberstyle=\small,
    stepnumber=1,
    numbersep=10pt,
    captionpos=t,
    escapeinside={\%*}{*)}
}

%%%----------%%%----------%%%----------%%%----------%%%


% Create a command to cleanly insert a snippet with the style above anywhere in the document
\newcommand{\insertcode}[2]{\begin{itemize}\item[]\lstinputlisting[caption=#2,label=#1,style=Style1]{#1}\end{itemize}} % The first argument is the script location/filename and the second is a caption for the listing


\begin{document}

\title{CYK Algorithm Implementation using Python}

\author{\large Mohammad Rahmdel - 9431305}

\date{\large Automata Languages: Dr.Saghiri}

\maketitle

% \section*{Python Code}


\begin{lstlisting}[caption=https://github.com/Mohammad-Rahmdel/cyk\_algorithm]

def read_input():
	file = open("input.txt","r") 
	x = []
	x = file.readlines()
	x = x[0]
	x = x[:-1] # ommiting '\n'
	file.close()
	return x

def read_grammer():
	file = open("Grammer.txt","r") 
	x = []
	x = file.readlines()
	for i in range(len(x)): # ommiting '\n'
		y = x[i]
		x[i] = y[:-1]
	file.close()
	return x

def parse_grammer(grammer):

	uppers = []     # storing rules containing terminals (e.g. A -> BC)
	lowers = []     # storing rules containing variables (e.g. B -> c)
	
	for rules in grammer :
		left, rights = rules.split('->', 1)   # find product rules
		left = left.strip()                   # removing extra spaces
		rights = rights.strip()
		right_split = rights.split('|')       # splitting rules 
		for right in right_split:
			right = right.strip()
			if right.isupper():               # separating variables and terminals
				uppers.append([left, right])
			else:
				lowers.append([left, right])
	return uppers, lowers
	
	
def CYK_Algorithm(bigs, littles, n):

	# creating a table for dynamic programming part
	matrix = [[set() for x in range(n)] for y in range(n)]  
	
	for i in range(n):
		char = inp[i]
		for x in littles:               # first row of the table
			if char == x[1]:
				matrix[0][i].add(x[0])
				
	for j in range(1, n):               # implementing CYK algorithm
		for k in range(n-j):  
			for l in range(j):
				B = matrix[l][k]
				C = matrix[j-l-1][k+l+1]
				X = set()
				for x in B:
					for y in C:
						X.add(x + y)
				
				for element in X:
					for x in bigs:
						if element == x[1]:
							matrix[j][k].add(x[0])
	return matrix

inp = read_input()
grammer = read_grammer()
uppers, lowers = parse_grammer(grammer)
matrix = CYK_Algorithm(uppers, lowers, len(inp))

# check whether last element in table contains starting symbol or not
if(matrix[len(inp)-1][0].__contains__('S') == True):        
	print("The input can be produced by the given grammer.")
else:
	print("The input cannot be produced by the given grammer!")

\end{lstlisting}



%\insertcode{"cyk.py"}{Nena would be proud.} % The first argument is the script location/filename and the second is a caption for the listing


\end{document}
